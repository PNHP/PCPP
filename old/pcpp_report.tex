\documentclass{article}
\usepackage{geometry}
\usepackage{fancyhdr}     %for headers,footers
\usepackage{underscore}  %needed if any text has underscores
\usepackage[T1]{fontenc} 
\usepackage[default]{gillius}
\usepackage[utf8]{inputenc} 
\setlength{\parskip}{1em}
\usepackage{lmodern}
\usepackage{newunicodechar}
\newunicodechar{<U+FB01>}{fi}
\newunicodechar{<U+FB00>}{ff}
\usepackage[singlelinecheck=false, justification=raggedright]{caption}
\usepackage{graphicx}
\graphicspath{ {E:/pcpp/PCPP/output/images/} }
\usepackage{longtable}
\usepackage[para]{threeparttablex}
\usepackage{multicol}
\usepackage{array}
\usepackage{multirow}
\usepackage{float}
\usepackage[skip=0.0\baselineskip,font=small]{caption}
\usepackage[usenames,dvipsnames,svgnames,table]{xcolor}
\usepackage{colortbl, xcolor}
\usepackage{wrapfig}
\usepackage{booktabs}
\usepackage{enumitem}
\usepackage{blindtext}
\usepackage{hyperref}
\urlstyle{same}


\title{Pennsylvania Plant Conservation Plan}
\date{2018\\ May}
\author{Christopher Tracey 
\and Molly Moore}

\newcolumntype{C}[1]{>{\centering\arraybackslash}p{#1}}
\newcolumntype{L}[1]{>{\raggedright\let\newline\\\arraybackslash\hspace{0pt}}p{#1}}

\geometry{letterpaper, portrait, top=0.45in, bottom=0.75in, left=0.5in, right=0.5in}
\pagestyle{fancy} \fancyhf{} \renewcommand\headrulewidth{0pt} %strip default header/footer stuff


%add footers
\cfoot{
  \small   %small font. The double slashes is newline in fancyhdr
   Pennsylvania Conservation Prioritorization Plan
}
\rfoot{page \thepage}
\begin{document}
\maketitle


\newpage
%Pennsylvania Plant Conservation Prioritorization Plan

%Pennsylvania Natural Heritage Program
%800 Waterfront Drive
%Pittsburgh, PA 15206

%This document should be cited as:
%\pagebreak


\tableofcontents
\newpage

\section{Project Goal}

\begin{wrapfigure}{R}{3in}
 \begin{center}
   \setlength{\fboxsep}{0pt}%
   \setlength{\fboxrule}{0.5pt}%
   \fbox{\includegraphics[width=3in]{Marshalliagrandifolia.jpg}}
 \end{center}
 \caption{Monongahela Barbara's buttons (\textit{Marshallia grandifolia}), is a globally rare plant that reaches the northern limits of its range in southwestern Pennslyvania.}
\end{wrapfigure}

\noindent\normalsize
\blindtext

\section{Introduction}
\noindent\normalsize
Two questions of conservation planning:
\begin{enumerate}
 \item Where conservation areas should be located? (systematic/spatial)
 \item What actions should be undertaken to manage them? (strategic)
\end{enumerate}

What data gaps exist for plants in Pennsylvania?

\subsection{State of Knowledge of the Pennsylvania Flora}

Floras\\
The most recent statewide flora is the "The Plants of Pennsylvania" (Ann Fowler Rhoads and Timothy A. Block. 2007).\\
Vascular Flora of Pennsylvania: Annotated Checklist and Atlas\\
"Flora of Pennsylvania" by Thomas Conrad Porter

\begin{wraptable}{R}{2.25in}
 \begin{center}
\begin{kframe}


{\ttfamily\noindent\bfseries\color{errorcolor}{\#\# Error in eval(expr, envir, enclos): object 'family\_X\_spp' not found}}\end{kframe}\caption{The ten largest plant families in Pennsylvania.}\label{table:kysymys}\begin{tabular}{ccc}\\\toprule\small\\\textbf{Family} & \textbf{Count}\\\midrule\begin{kframe}

{\ttfamily\noindent\bfseries\color{errorcolor}{\#\# Error in nrow(family\_X\_spp): object 'family\_X\_spp' not found}}\end{kframe}\bottomrule\end{tabular}







